\chapter{Практический раздел}

    \section{Задание 1. Знакомство с программой PCLAB}
        
        \subsection*{Идентификационная информация на вкладке «Идентификация процессора».}
        
\noindent
Vendor ID: "GenuineIntel"; CPUID level 13\newline

\noindent
Дополнительные функции Intel:\newline
Верисия 0001067a:\newline
Type 0 - Original OEM\newline
Family 6 - Pentium Pro\newline
Model 7 - Pentium III/Pentium III Xeon - external L2 cache\newline
Stepping 10\newline
Reserved 4\newline

\noindent
Extended brand string: "Intel(R) Celeron(R) CPU E3300 @ 2.50GHz"\newline
CLFLUSH instruction cache line size: 8\newline
Initial APIC ID: 1\newline
Hyper threading siblings: 2\newline

\noindent
Feature flags bfebfbff:\newline
0 FPU Присутствует Математический сопроцессор\newline
1 VME Поддержка расширенных возможностей обработки прерываний в режиме виртуального i8086\newline
2 DE Поддержка отладки\newline
3 PSE Поддержка страниц размером 4 MB\newline
4 TSC Счетчик меток реального времени\newline
5 MSR Поддержка команд rdmsr и wrmsr\newline
6 PAE Поддержка физического адреса более 32 бит\newline
7 MCE Поддержка исключений 18 - об аппаратных ошибках\newline
8 CX8 Поддержка инструкции cmpxchg8b\newline
9 APIC Микропроцессор содержит программно доступный контроллер прерываний\newline
11 SEP Поддержка инструкций быстрых системных вызовов sysenter и sysexit\newline
12 MTRR Поддержка регистра mtrr\_cap (относится к MSR-регистрам)\newline
13 PGE Поддержка глобальных страниц\newline
14 MCA Поддержка архитектуры машинного контроля\newline
15 CMOV Поддержка инструкций условной пересылки cmov, fcmovcc, fcomi\newline
16 PAT Процессор поддерживает таблицу атрибутов страницы\newline
17 PSE-36 Процессор поддерживает 4 MB страницы, которые способны адресовать физическую память до 64 GB\newline
19 CLFLSH Поддержка инструкции CLFLUSH\newline
21 DS Поддержка записи отладочной информации\newline
22 ACPI Управление охлаждением процессора с помощью пустых циклов в зависимости от температуры\newline
23 MMX Поддержка MMX\newline
24 FXSR Поддержка инструкций FXSAVE и FXRSTOR\newline
25 SSE Поддержка SSE\newline
26 SSE2 Поддержка SSE2\newline
27 SS Управление конфликтующими типами памяти\newline
28 HTT Поддержка Hyper-Threading\newline
29 TM Поддержка автоматического мониторинга температуры\newline
31 SBF Сигнал Останова при FERR\newline

\noindent
TLB and cache info:\newline
b1: unknown TLB/cache descriptor\newline
b0: дескриптор TLB-команд, 4K страницы, асс. 4-направ., 128 элементов\newline
05: unknown TLB/cache descriptor\newline
f0: unknown TLB/cache descriptor\newline
57: unknown TLB/cache descriptor\newline
56: unknown TLB/cache descriptor\newline
78: unknown TLB/cache descriptor\newline
30: L1 кэш-команд, 32 KB, асс. 8-направ., длина строки 64 байта\newline
b4: unknown TLB/cache descriptor\newline
2c: L1 кэш-данных, 32 KB, асс. 8-направ., длина строки 64 байта\newline
Processor serial: 0001-067A-BFEB-FBFF-0400-E3BD\newline
    

    \section{Задание 2. Определение параметров процессора}
    
        \noindent
        Размер линейки кэш-памяти верхнего уровня = 32 KB
    
        \noindent
        Oбъем физической памяти до 64 GB.
    
    \newpage
    
    \section{Задание 3. Эксперимент <<Исследование расслоения динамической памяти>>}

        \textbf{Цель   эксперимента:} определение   способа   трансляции   физического   адреса, используемого при обращении к динамической памяти.   
        
        \textbf{Исходные данные:} размер линейки кэш-памяти верхнего уровня; объем физической памяти
        
        \imgw{task_03}{h!}{16cm}{Результат эксперимента задания 3}
        
        \[
            \begin{aligned}
                &\text{Т}_1 = 128\\
                &\text{Т}_2 = 4096\\
                &\text{П} = 64\\
                &\text{Б} = \frac{\text{Т}_1}{\text{П}} = 2\\
                &\text{PC} = \frac{\text{T}_2}{\text{Б}} = 2048\\
                &\text{С} = \frac{\text{О}}{\text{РС} \cdot \text{Б} \cdot \text{П}} = \frac{4\text{Гб}}{2048 * 2 * 32 \text{Кб}}= 32 \text{ страницы}\\
            \end{aligned}
        \]
        
        \subsection*{Результат}
        
            Количество страниц физической памяти - 32.
        
    \newpage
        
    \section{Задание 4. Эксперимент <<Сравнение эффективности ссылочных и векторных структур>>}
    
        \textbf{Цель   эксперимента:} оценка   влияния   зависимости   команд   по   данным   на эффективность вычислений. 
        
        \imgw{task_04}{h!}{16cm}{Результат эксперимента задания 4}
        
        \subsection*{Результат}
        
        Список обрабатывался в 19,804851 раз дольше.
        
    \newpage
        
    \section{Задание 5. Эксперимент <<Исследование эффективности программной предвыборки>>}
    
        \textbf{Цель   эксперимента:} выявление   способов   ускорения   вычислений   благодаря применению предвыборки данных.
        
        \textbf{Исходные данные:} степень ассоциативности и размер TLB данных.
        
        \imgw{task_05}{h!}{16cm}{Результат эксперимента задания 5}
        
        \subsection*{Результат}
        
        Обработка без загрузки таблицы страниц в TLB производилась в 1,3134548 раз дольше.
        
    \newpage
    
    \section{Задание 6. Эксперимент <<Исследование способов эффективного  чтения оперативной памяти>>}
    
        \textbf{Цель эксперимента:}  исследование возможности ускорения вычислений благодаря использованию структур данных, оптимизирующих механизм чтения оперативной памяти.
        
        \textbf{Исходные   данные:}  Адресное расстояние между банками памяти, размер буфера чтения.
        
        \imgw{task_06}{h!}{16cm}{Результат эксперимента задания 6}
        
        \subsection*{Результат}
        
            Неоптимизированная структура обрабатывалась в 1,6404477 раз дольше.
            
    \newpage
        
    \section{Задание 7. Эксперимент <<Исследование конфликтов в кэш-памяти>>}
    
        \textbf{Цель   эксперимента:}  исследование   влияния   конфликтов   кэш-памяти   на эффективность вычислений.
        
        \textbf{Исходные   данные:}  Размер банка кэш-памяти данных первого и второго уровня, степень ассоциативности кэш-памяти первого и второго уровня, размер линейки кэш- памяти первого и второго уровня.
        
        \imgw{task_07}{h!}{16cm}{Результат эксперимента задания 7}
        
        \subsection*{Результат}
        
            Чтение с конфликтами банков производилось в 6,1521271 раз дольше.
    
    \newpage
    
    \section{Задание 8. Эксперимент <<Сравнение алгоритмов сортировки>>}
    
        \textbf{Цель эксперимента:} исследование способов эффективного использования  памяти и выявление наиболее эффективных алгоритмов сортировки, применимых в вычислительных системах.
        
        \textbf{Исходные данные:} количество процессоров вычислительной системы, размер пакета, количество элементов в массиве, разрядность элементов массива
        
        \imgw{task_08}{h!}{16cm}{Результат эксперимента задания 8}
        
        \subsection*{Результат}
            
            QuickSort работал в 1,9521806 раз дольше Radix-Counting Sort.
            
            QuickSort работал в 2,2885507 раз дольше Radix-Counting Sort, оптимизированного под 8-процессорную ЭВМ.