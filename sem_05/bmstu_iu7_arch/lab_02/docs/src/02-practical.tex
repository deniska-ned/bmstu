\chapter{Практический раздел}

    \section{Задание №1}
    
        \subsection{Исходный код}
        
            \listingfile{hha.s}{}{Исходный код программы}{}
            
\newpage
        
        \subsection{Псевдокод}
        
            \listingfile{hha.c}{c}{Псевдокод}{}
            
\newpage

        \subsection{Дизассемблированный листинг}

            \listingfile{hha_disasm.s}{}{Дизассемблированный листинг}{}

\newpage

    \section{Задание №2}

        Получить снимок экрана, содержащий временную диаграмму выполнения стадий выборки и диспетчеризации команды с указанным адресом. 80000014, 2-я.

        \imgw{task_02.png}{htb!}{\textwidth}{Стадии выборки и диспечеризации команды по 80000014 адреса на 2ой итерации}

        \begin{itemize}
            \item Такт 29 - происходит выборка команды с адресом 80000014
            \item Такт 30 - команда попадает на вход блока управления метаинформацией; записывается в \(pc\_table\) с \(pc\_id = 1\).
        \end{itemize}

\newpage

    \section{Задание №3}

        Получить снимок экрана, содержащий временную диаграмму выполнения стадии декодирования и планирования на выполнение команды с указанным адресом. 80000020, 2-я.

        \imgw{task_03.png}{htb!}{\textwidth}{Стадии декодирования и планирования на выполнение команды 80000020 на 2ой итерации}
        
        \begin{itemize}
            \item Такт 32 - выборка команды с адресом 80000020
            \item Такт 33 - диспечеризация. Команда попадает на вход блока управления метаинформацией; записывается в \(pc\_table\) с \(pc\_id = 4\).
            \item Такт 34-38 - ожидание
            \item Такт 39 - планирование на выполнение
        \end{itemize}
        
        Затем 2 такта пребывает в конфликте \(rs2\_conflict\).

\newpage

    \section{Задание №4}
    
        Получить снимок экрана, содержащий временную диаграмму выполнения стадии выполнения команды с указанным адресом. 8000000с, 2-я. (lw x2,0(x1))
        
        \imgw{task_04.png}{htb!}{\textwidth}{Снимок экрана, содержащий временную диаграмму стадии выполнения команды по адресу 800000с на второй итерации. (lw x2,0(x1)) }

\newpage

    \section{Задание №5}
    
        Адрес команды: 80000018
        
        \imgw{task_05_hha_01.png}{htb!}{\textwidth}{Выборка, диспечеризация}
        \imgw{task_05_hha_02.png}{htb!}{\textwidth}{Планирование, выполнение}
        
        \subsection{Оптимизация}
        
            Можно оптимизировать этот участок:
            
            \listingfile{origin.s}{}{}{}
            
            Каждая следующая команда add зависит от предыдущей команды загрузки, из-за чего они простаивают ожидая разрешение конфликта. Это можно оптимизировать путем совместного запроса данных:
            
            \listingfile{optimized.s}{}{}{}
            

            \imgw{trassirovka_org.png}{htb!}{\textwidth}{Оригинальная программа}
            
            \imgw{trassirovka_opt.png}{htb!}{\textwidth}{Оптимизированная программа}